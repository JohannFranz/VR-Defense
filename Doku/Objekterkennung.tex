%%%%%%%%%%%%%%%%%%% Objekterkennung.tex %%%%%%%%%%%%%%%%%%%%%%%%%%%%%
%
% LaTeX-Vorlage zur Erstellung von Projekt-Dokumentationen
% im Fachbereich Informatik der Hochschule Trier
%
% Basis: Vorlage svmono des Springer Verlags
%
%%%%%%%%%%%%%%%%%%%%%%%%%%%%%%%%%%%%%%%%%%%%%%%%%%%%%%%%%%%%%

\documentclass[envcountsame,envcountchap, deutsch]{i-studis}

\usepackage{makeidx}         	% Index
\usepackage{multicol}        	% Zweispaltiger Index
%\usepackage[bottom]{footmisc}	% Erzeugung von Fu�noten

%%-----------------------------------------------------
%\newif\ifpdf
%\ifx\pdfoutput\undefined
%\pdffalse
%\else
%\pdfoutput=1
%\pdftrue
%\fi
%%--------------------------------------------------------
%\ifpdf
\usepackage[pdftex]{graphicx}
\usepackage[pdftex,plainpages=false]{hyperref}
%\else
%\usepackage{graphicx}
%\usepackage[plainpages=false]{hyperref}
%\fi

%%-----------------------------------------------------
\usepackage{color}				% Farbverwaltung
%\usepackage{ngerman} 			% Neue deutsche Rechtsschreibung
\usepackage[english, ngerman]{babel}
\usepackage[latin1]{inputenc} 	% Erm�glicht Umlaute-Darstellung
%\usepackage[utf8]{inputenc}  	% Erm�glicht Umlaute-Darstellung unter Linux (je nach verwendetem Format)
\usepackage{caption}
\newcommand{\source}[1]{\caption*{Quelle: {#1}} }

%-----------------------------------------------------
\usepackage{listings} 			% Code-Darstellung
\lstset
{
	basicstyle=\scriptsize, 	% print whole listing small
	keywordstyle=\color{blue}\bfseries,
								% underlined bold black keywords
	identifierstyle=, 			% nothing happens
	commentstyle=\color{red}, 	% white comments
	stringstyle=\ttfamily, 		% typewriter type for strings
	showstringspaces=false, 	% no special string spaces
	framexleftmargin=7mm, 
	tabsize=3,
	showtabs=false,
	frame=single, 
	rulesepcolor=\color{blue},
	numbers=left,
	linewidth=146mm,
	xleftmargin=8mm
}
\usepackage{textcomp} 			% Celsius-Darstellung
\usepackage{amssymb,amsfonts,amstext,amsmath,bbm}	% Mathematische Symbole
\usepackage[german, ruled, vlined]{algorithm2e}
\usepackage[a4paper]{geometry} % Andere Formatierung
\usepackage{bibgerm}
\usepackage{array}
\hyphenation{Ele-men-tar-ob-jek-te  ab-ge-tas-tet Aus-wer-tung House-holder-Matrix Le-ast-Squa-res-Al-go-ri-th-men} 		% Weitere Silbentrennung bei Bedarf angeben
\setlength{\textheight}{1.1\textheight}
\pagestyle{myheadings} 			% Erzeugt selbstdefinierte Kopfzeile
\makeindex 						% Index-Erstellung


%--------------------------------------------------------------------------
\begin{document}
%------------------------- Titelblatt -------------------------------------
\title{VR-Defense}
\project{Dokumentation zum Fach Projektstudium}
%--------------------------------------------------------------------------
\supervisor{Professor Dr. Christof Rezk-Salama} 		% Betreuer der Arbeit
\author{Johann Franz}							% Autor der Arbeit

\address{Trier,} 							% Im Zusammenhang mit dem Datum wird hinter dem Ort ein Komma angegeben
\submitdate{10.10.2020} 				% Abgabedatum
%\begingroup
%  \renewcommand{\thepage}{title}
%  \mytitlepage
%  \newpage
%\endgroup
\begingroup
  \renewcommand{\thepage}{Titel}
  \mytitlepage
  \newpage
\endgroup
%--------------------------------------------------------------------------
\frontmatter 
%--------------------------------------------------------------------------
%\kurzfassung

%% deutsch
\paragraph*{}
In dieser Arbeit wird mit You Only Look Once (YOLO) ein Objekterkennungsverfahren aus dem Deep Learning vorgestellt. 
Dank der Verwendung einer einheitlichen Architektur ist YOLO in der Lage ein Bild parallelisiert in einem Netzdurchlauf zu analysieren. \newline
Die Evaluierung der Leistungsf�higkeit des Netzes erfolgt durch die Mean Average Precision-Metrik. 
Im Vergleich mit schnellen Verfahren zeigt sich eine mehr als doppelte Mean Average Precision.
Dabei schafft es YOLO konstant mehr als 30 Bilder pro Sekunde zu verarbeiten. 



 			% Kurzfassung Deutsch/English
\tableofcontents 						% Inhaltsverzeichnis
%--------------------------------------------------------------------------
\mainmatter                        		% Hauptteil (ab hier arab. Seitenzahlen)
%--------------------------------------------------------------------------
% Die Kapitel werden in separaten .tex-Dateien abgelegt und hier eingebunden.
%\chapter{Einleitung und Problemstellung}

Das Maschinelle Sehen besch{\"a}ftigt sich mit der computergest{\"u}tzten Analyse einer Szene und der darin enthaltenen Objekte. 
Bis zum heutigen Zeitpunkt gilt diese Aufgabenstellung als die schwierigste aller visuellen Aufgaben, die einem Computer zugeteilt werden kann \cite[S. 577]{Szeliski2010}.
Die Gr{\"u}nde hierf{\"u}r liegen einerseits in einer riesigen Menge unterschiedlicher Objektkategorien, andererseits in einer F{\"u}lle von Vielf{\"a}ltigkeiten innerhalb jeder Kategorie \cite[S. 577]{Szeliski2010}. 
So existieren zum Beispiel Hunde in zahlreichen Rassen, Gr{\"o}\ss{en} und Altersgruppen. 
\newline
Die Objekterkennung ist ein Teilfeld des Maschinellen Sehens. 
Ziel ist die Klassifizierung und Lokalisierung von Objekten in Bildern \cite[S. 577]{Szeliski2010}. 
Dazu wird ein erkanntes Objekt mit einer Bounding Box umh{\"u}llt und beschriftet. 
Damit einzelne Verfahren verglichen werden k{\"o}nnen, werden Bilddatens{\"a}tze mit Annotationen zur Verf{\"u}gung gestellt. In j{\"a}hrlichen Challenges messen sich die besten Algorithmen \cite{Everingham2010}\cite{Russakovsky2015}. 
F{\"u}r einen aussagekr{\"a}ftigen Vergleich ben{\"o}tigt es definierter Qualit{\"a}ts- und Messkriterien. 
F{\"u}r jedes erkannte Objekt wird eine Klassifizierung und Lokalisierung vorgenommen und anschlie\ss{end} jeweils ein Zuversichtswert angegeben.
Dieser Wert belegt wie sicher sich das Verfahren ist, dass das Objekt der entsprechenden Klasse angeh{\"o}rt bzw. von einer passenden Box umschlossen ist. 
Mit der Average-Precision (AP) wurde f{\"u}r die VOC Challenge im Jahr 2007 ein Evaluierungsverfahren eingef{\"u}hrt, das bereits in der Information Retrieval Verwendung findet \cite{Everingham2010}\cite[S. 160]{Manning2008}. 
F{\"u}r jede Klasse wird die Average Precision bestimmt und schlie\ss{lich} in einem Mittelwert  zusammengefasst. \newline
Seit 2012 hat die reine Klassifizierung der Objekte mit der Verwendung von Deep Learning Verfahren, wie dem AlexNet, wesentliche Fortschritte gemacht \cite{Krizhevsky2012}\cite{He2015}\cite{Szegedy2014}. 
Dies hat zu einem Umdenken in der Forschung gef{\"u}hrt. 
Neue Objekterkennungsverfahren partitionieren das Bild in Kachel und schicken diese anschlie\ss{end} durch ein Klassifizierungsnetz \cite{Sermanet2014}. 
Alternativ dazu bestimmt R-CNN in einem Vorverarbeitungsschritt Regionen, in denen Objekte zu erwarten sind. 
Diese Regionen werden anschlie\ss{end} klassifiziert \cite{Girshick2013}. \newline
Probleme dieser Verfahren sind sowohl die Performanz, es kann mehrere Sekunden dauern bis ein Bild verarbeitet wurde. 
Andererseits ben{\"o}tigen genannte Verfahren mehrere Arbeitsschritte pro Bild \cite{Redmon2015}. 
Zu diesem Zweck wurde You Only Look Once (YOLO) entwickelt. 
YOLO interpretiert die Objekterkennung \glqq als ein einzelnes Regressionsproblem, direkt aus Bildpixel in Bounding Box-Koordinaten und  Klassenwahrscheinlichkeiten\grqq{} \cite{Redmon2015}. 
Das erm{\"o}glicht Training und Einsatz eines neuronalen Netzes mit einer einheitlichen Architektur. 
YOLO ist so in der Lage Bilder in Echtzeit\footnote{Mindestens 30 Bilder pro Sekunde} zu analysieren.
Der Trainingsprozess verallgemeinert die Objektinformationen soweit, dass sogar Kunst besser erkannt wird als bei alternativen Verfahren.
Im Gegenzug hat YOLO verst{\"a}rkt Probleme mit kleinen Objekten \cite{Redmon2015}.
Inzwischen wurde bereits die dritte Version von YOLO ver{\"o}ffentlicht \cite{Redmon2018}. 
In dieser Arbeit wird die initiale Version untersucht und vorgestellt.
\chapter{Spielkonzept}
VR-Defense ist ein kartenbasiertes Tower-Defense Spiel, das sowohl in Virtual Reality als auch im normalen Modus gespielt werden kann. 
Der Spieler erhält in jeder Mission ein Deck aus 10 Karten (Mercenaries). 
Vor Beginn des Spiels wird das Deck gemischt. 
Teil des Decks sind drei normale Einheitentypen und eine Spezialeinheit.
Die Spezialeinheit ist nur einmal im Deck vorhanden.
Die drei normalen Einheitentypen sind aufgeteilt auf 2 Einheiten mit ballistischen Waffen und eine Einheit mit einem Flammenwerfer. 
Strategische Diversität der Einheiten wird durch unterschiedliche Lebens- und Schadenswerte erreicht.
Ziel des Spiels ist es zu überleben bis alle Wellen abgeschlossen sind. 
Dazu muss der Spieler am Ende mindestens ein Leben übrig haben.

\section{Formation}
In jeder Welle können beliebig viele Gegner (Minions) erzeugt werden. 
Damit die Kämpfe geordnet stattfinden können, treten die Minions in Formation an.
Es greift stets nur die erste Reihe der Minions die Mercenaries an. 
Sobald ein Minion stirbt, rückt der nächstmögliche Minion nach und beginnt mit dem Kampf.
Der Einsatz von Formationen ermöglicht eine bessere Planung für den Spieler. 
Zu keinem Zeitpunkt besteht die Chance, dass die Mercenaries umzingelt werden.

\section{Karten}
Zu Beginn jeder Runde muss der Spieler drei Karten auf seiner Hand haben. 
Dazu werden bis zu drei Karten ausgeteilt.
Falls der Spieler keine der Karten ausspielt, bleiben sie bis zur nächsten Runde erhalten.
In diesem Fall werden keine neuen Karten gezogen.
Eine Karte enthält zusätzlich zum Mercenary-Typ weitere Informationen über den Schaden, Leben und Reichweite der Einheit.

\section{Platzierung}
Die Platzierung der Mercenaries erfolgt auf einem sogenannten Placement. 
Jedes Placement gehört zu einer Placement-Areas. 
Pro Placement-Area können maximal drei Placements hinzugefügt werden.
Eine gezielte Platzierung der Areas erzwingt strategische Entscheidungen auf Seiten des Spielers.

\subsection{Lane Placement}
Bei einem Lane Placement werden auf einem Weg mehrere Placement-Areas verteilt.
Dies ermöglicht die Aufteilung der Mercenaries in Stopper und Sniper. 
Die Aufgabe der Stopper ist es einen Bossgegner zu schwächen. 
Wohingegen ein Sniper mit wenig Leben aber viel Durchschlagskraft den Boss besiegt.

\subsection{Parallel Placement}
Die Platzierung der Areas auf parallelen Bahnen zwingt den Spieler zu Beginn eine Bahn zu favorisieren oder beide zu vernachlässigen. 
Im Laufe des Spiels wird dieser Nachteil jedoch ausgeglichen.

\section{Unterschiede zwischen den Modi}
Während im normalen Modus eine isometrische Perspektive gewählt wurde, ist der Spieler in der virtuellen Ansicht direkt im Geschehen.
Die isometrische Ansicht bietet eine strategische Übersicht, die schnelle Entscheidungen ermöglicht.
Im direkten Gegensatz dazu muss der Spieler im virtuellen Modus erst die Welt erkunden und herausfinden wo die einzelnen Spawnpunkte bzw. Pfade liegen. 


%\chapter{LaTeX-Bausteine}\label{Stile}

Der Text wird in bis zu drei Ebenen gegliedert:

\begin{enumerate}
  \item Kapitel ( \verb \chapter{Kapitel} ), \index{Kapitel}
  \item Unterkapitel  ( \verb \section{Abschnitt} ) und
  \item Unterunterkapitel  ( \verb \subsection{Unterabschnitte} ).
\end{enumerate}

\section{Abschnitt}\index{Abschnitt}
Text der Gliederungsebene 2.


\subsection{Unterabschnitt} \index{Unterabschnitt}
Text der Gliederungsebene 3.
Text Text Text Text Text Text Text Text Text Text Text Text Text Text Text
Beispiel f�r Quelltext\index{Quelltext} \\[2 ex]
\noindent
\begin{minipage}{1.0\textwidth} \small
\begin{lstlisting}
	Prozess 1:
	
	Acquire();
		a := 1;
	Release();
	...
	Acquire();
	if(b == 0)
	{					
		c := 3;
		d := a;
	}				
	Release();
\end{lstlisting}
\end{minipage}

\vspace{2cm}
\noindent
\begin{minipage}{1.0\textwidth} \small
\begin{lstlisting}
	Prozess 2:
	
	Acquire();
		b := 1;
	Release();
	...
	Acquire();
	if(a == 0)
	{					
		c := 5;
		d := b;
	}				
	Release();
\end{lstlisting}
\end{minipage}
\vskip 1em

Gr��ere Code-Fragmente sollten im Anhang eingef�gt werden.

\section{Abbildungen und Tabellen}

Abbildung\index{Abbildung} und Tabellen\index{Tabelle} werden zentriert eingef�gt. Grunds�tzlich sollen sie
erst dann erscheinen, nach dem sie im Text angesprochen wurden (siehe Abb. \ref{a1}). Abbildungen und Tabellen (siehe Tabelle \ref{t1}) k�nnen
im (flie�enden) Text (\verb here ), am Seitenanfang (\verb top ), am Seitenende
(\verb bottom ) oder auch gesammelt auf einer nachfolgenden Seite (\verb page )
oder auch ganz am Ende der Ausarbeitung erscheinen. Letzteres sollte man nur
dann w�hlen, wenn die Bilder g�nstig zusammen zu betrachten sind und die
Ausarbeitung nicht zu lang ist ($< 20$ Seiten).

\begin{figure} %[hbtp]
	\centering
		\includegraphics{images/p1ReadSeq.pdf}
	\caption{Bezeichnung der Abbildung}
	\label{a1}
\end{figure}

\begin{table} %[hbtp]
	\centering
		\begin{tabular}{l | l l l l}
		\textbf{Prozesse} & \textbf{Zeit} $\rightarrow$ \\
		\hline
			$P_{1}$ & $W(x)1$ \\
			$P_{2}$ & & $W(x)2$ \\
			$P_{3}$ & & $R(x)2$ & & $R(x)1$\\
			$P_{4}$ & & & $R(x)2$ & $R(x)1$\\
		\end{tabular}
	\caption{Bezeichnung der Tabelle}
	\label{t1}
\end{table}


\section{Mathematische Formel}\index{Formel}
Mathematische Formeln bzw. Formulierungen k�nnen sowohl im
laufenden Text (z.B. $y=x^2$) oder abgesetzt und zentriert im Text
erscheinen. Gleichungen sollten f�r Referenzierungen nummeriert
werden (siehe Formel \ref{gl-1}).
\begin{equation}
\label{gl-1}
e_{i}=\sum _{i=1}^{n}w_{i}x_{i}
\end{equation}

Entscheidungsformel:

\begin{equation}
\psi(t)=\left\{\begin{array}{ccc}
1 &  \qquad 0 <= t < \frac{1}{2} \\
-1 &  \qquad \frac{1}{2} <= t <1 \\
0 & \qquad sonst
\end{array} \right.
\end{equation}


Matrix:\index{Matrix}
\begin{equation}
A = \left(
\begin{array}{llll}
a_{11} & a_{12} & \ldots & a_{1n} \\
a_{21} & a_{22} & \ldots & a_{2n} \\
\vdots & \vdots & \ddots & \vdots \\
a_{n1} & a_{n2} & \ldots & a_{nn} \\
\end{array}
\right)
\end{equation}

Vektor:\index{Vektor} 

\begin{equation}
\overline{a} = \left(
\begin{array}{c}
a_{1}\\
a_{2}\\
\vdots\\
a_{n}\\
\end{array}
\right)
\end{equation}

\section{S�tze, Lemmas und Definitionen}\index{Satz}\index{Lemma}\index{Definition}

S�tze, Lemmas, Definitionen, Beweise,\index{Beweis} Beispiele\index{Beispiel} k�nnen in speziell daf�r vorgesehenen Umgebungen erstellt werden.

\begin{definition}(Optimierungsproblem)

Ein \emph{Optimierungsproblem} $\mathcal{P}$ ist festgelegt durch ein Tupel
$(I_\mathcal{P}, sol_\mathcal{P}, m_\mathcal{P}, goal)$ wobei gilt

\begin{enumerate}
\item $I_\mathcal{P}$ ist die Menge der Instanzen,
\item $sol_\mathcal{P} : I_\mathcal{P} \longmapsto \mathbb{P}(S_\mathcal{P})$ ist eine Funktion, die jeder Instanz $x \in I_\mathcal{P}$ eine Menge zul�ssiger L�sungen zuweist,
\item $m_\mathcal{P} : I_\mathcal{P} \times S_\mathcal{P} \longmapsto \mathbb{N}$ ist eine Funktion, die jedem Paar $(x,y(x))$ mit $x \in I_\mathcal{P}$ und $y(x) \in sol_\mathcal{P}(x)$ eine
Zahl $m_\mathcal{P}(x,y(x)) \in \mathbb{N}$ zuordnet (= Ma� f�r die L�sung $y(x)$ der Instanz $x$), und
\item $goal \in \{min,max\}$.
\end{enumerate}

\end{definition}

\begin{example} MINIMUM TRAVELING SALESMAN (MIN-TSP)
\begin{itemize}
\item $I_{MIN-TSP} =_{def}$ s.o., ebenso $S_{MIN-TSP}$
\item $sol_{MIN-TSP}(m,D) =_{def} S_{MIN-TSP} \cap \mathbb{N}^m$ 
\item $m_{MIN-TSP}((m,D),(c_1, \ldots , c_m)) =_{def} \sum_{i=1}^{m-1} D(c_i, c_{i+1}) + D(c_m,c_1)$ 
\item $goal_{MIN-TSP} =_{def} min$
\end{itemize}
\begin{flushright}
$\qed$
\end{flushright}
\end{example}

\begin{theorem} Sei $\mathcal{P}$ ein \textbf{NP}-hartes Optimierungsproblem.
Wenn $\mathcal{P} \in$ \textbf{PO}, dann ist \textbf{P} = \textbf{NP}.
\end{theorem}

\begin{proof} Um zu zeigen, dass \textbf{P} = \textbf{NP} gilt, gen�gt es
wegen Satz A.30 zu zeigen, dass ein einziges \textbf{NP}-vollst�ndiges
Problem in \textbf{P} liegt. Sei also $\mathcal{P}'$ ein beliebiges \textbf{NP}-vollst�ndiges Problem.

Weil $\mathcal{P}$ nach Voraussetzung \textbf{NP}-hart ist, gilt insbesondere
$\mathcal{P}' \leq_T \mathcal{P}_C$. Sei $R$ der zugeh�rige
Polynomialzeit-Algorithmus dieser Turing-Reduktion.
Weiter ist $\mathcal{P} \in$ \textbf{PO} vorausgesetzt, etwa verm�ge eines
Polynomialzeit-Algorithmus $A$. Aus den beiden
Polynomialzeit-Algorithmen $R$ und $A$ erh�lt man nun
leicht einen effizienten Algorithmus f�r $\mathcal{P}'$: Ersetzt man
in $R$ das Orakel durch $A$, ergibt dies insgesamt eine polynomielle
Laufzeit. 
%\begin{flushright}
$\qed$
% \end{flushright}
\end{proof}

\begin{lemma} Aus \textbf{PO} $=$ \textbf{NPO} folgt \textbf{P} $=$ \textbf{NP}.
\end{lemma}

\begin{proof} Es gen�gt zu zeigen, dass unter der angegeben
Voraussetzung KNAPSACK $\in$ \textbf{P} ist.

Nach Voraussetung ist MAXIMUM KNAPSACK $\in$ \textbf{PO},
d.h. die Berechnung von $m^*(x)$ f�r jede Instanz $x$ ist
in Polynomialzeit m�glich. Um KNAPSACK bei Eingabe
$(x,k)$ zu entscheiden, m�ssen wir nur noch $m^*(x) \geq k$
pr�fen. Ist das der Fall, geben wir $1$, sonst $0$ aus. Dies
bleibt insgesamt ein Polynomialzeit-Algorithmus. 
\begin{flushright}
$\qed$
\end{flushright}
\end{proof}

\section{Fu�noten}

In einer Fu�note k�nnen erg�nzende Informationen\footnote{Informationen die f�r die Arbeit zweitrangig sind, jedoch f�r den Leser interessant sein k�nnten.} angegeben werden. Au�erdem kann eine Fu�note auch Links enthalten. Wird in der Arbeit eine Software (zum Beispiel Java-API\footnote{\url{http://java.sun.com/}}) eingesetzt, so kann die Quelle, die diese Software zur Verf�gung stellt in der Fu�note angegeben werden.

\section{Literaturverweise}\index{Literatur}
Alle benutzte Literatur wird im Literaturverzeichnis angegeben\footnote{Dazu wird ein sogennanter bib-File, literatur.bib verwendet.}. Alle angegebene Literatur sollte mindestens einmal im Text referenziert werden\cite{Coulouris:02}.
%\chapter{Zusammenfassung und Ausblick}
Diese Arbeit hat mit You Only Look Once einen Deep Learning Algorithmus f�r die Objekterkennung vorgestellt. 
Die Leistungsf�higkeit von YOLO wird auf Basis eines bin�ren Klassifikators evaluiert. 
Dazu werden relative Kenngr��en in Form der Genauigkeit und Trefferquote bestimmt und in einem Graph dargestellt.
Die Fl�che unterhalb des Graphen ist als Average Precision definiert und wird in einem Mittelwert �ber alle Klassen in der Mean Average Precision zusammengefasst.\newline
Kontr�r zu alternativen Verfahren interpretiert YOLO das Bild als ein Regressions- statt einem Klassifikationsproblem.
YOLO verwendet eine einheitliche Netzarchitektur, die eine Parallelisierung der Prozesse erlaubt.
Daraus ergibt sich eine Verarbeitungsgeschwindigkeit von mehr als 30 Bildern pro Sekunde. 
Das Bild wird in Zellen aufgeteilt und jede Zelle bestimmt eine feste Anzahl von Bounding Boxes. \newline
Im Vergleich mit alternativen Echtzeitverfahren zeigt sich, dass YOLO eine mehr als doppelt so hohe mAP erreicht. 
Eine schlanke Version von YOLO, genannt Fast YOLO, kann 50\% mehr Bilder pro Sekunde verarbeiten als das schnellste alternative Verfahren.
Bei der Fehleranalyse zeigt sich, dass YOLO bei der Lokalisierung schlechter als die Konkurrenz abschneidet.
% ...
%--------------------------------------------------------------------------
\backmatter                        		% Anhang
%-------------------------------------------------------------------------
\bibliographystyle{geralpha}			% Literaturverzeichnis
\bibliography{literatur}     			% BibTeX-File literatur.bib
%--------------------------------------------------------------------------
\printindex 							% Index (optional)
%--------------------------------------------------------------------------
%\begin{appendix}						% Anh�nge sind i.d.R. optional
%   \chapter{Glossar}

			% Glossar   
%\end{appendix}

\end{document}
