\chapter{Spielkonzept}
VR-Defense ist ein kartenbasiertes Tower-Defense Spiel, das sowohl in Virtual Reality als auch im normalen Modus gespielt werden kann. 
Der Spieler erhält in jeder Mission ein Deck aus 10 Karten (Mercenaries). 
Vor Beginn des Spiels wird das Deck gemischt. 
Teil des Decks sind drei normale Einheitentypen und eine Spezialeinheit.
Die Spezialeinheit ist nur einmal im Deck vorhanden.
Die drei normalen Einheitentypen sind aufgeteilt auf 2 Einheiten mit ballistischen Waffen und eine Einheit mit einem Flammenwerfer. 
Strategische Diversität der Einheiten wird durch unterschiedliche Lebens- und Schadenswerte erreicht.
Ziel des Spiels ist es zu überleben bis alle Wellen abgeschlossen sind. 
Dazu muss der Spieler am Ende mindestens ein Leben übrig haben.

\section{Formation}
In jeder Welle können beliebig viele Gegner (Minions) erzeugt werden. 
Damit die Kämpfe geordnet stattfinden können, treten die Minions in Formation an.
Es greift stets nur die erste Reihe der Minions die Mercenaries an. 
Sobald ein Minion stirbt, rückt der nächstmögliche Minion nach und beginnt mit dem Kampf.
Der Einsatz von Formationen ermöglicht eine bessere Planung für den Spieler. 
Zu keinem Zeitpunkt besteht die Chance, dass die Mercenaries umzingelt werden.

\section{Karten}
Zu Beginn jeder Runde muss der Spieler drei Karten auf seiner Hand haben. 
Dazu werden bis zu drei Karten ausgeteilt.
Falls der Spieler keine der Karten ausspielt, bleiben sie bis zur nächsten Runde erhalten.
In diesem Fall werden keine neuen Karten gezogen.
Eine Karte enthält zusätzlich zum Mercenary-Typ weitere Informationen über den Schaden, Leben und Reichweite der Einheit.

\section{Platzierung}
Die Platzierung der Mercenaries erfolgt auf einem sogenannten Placement. 
Jedes Placement gehört zu einer Placement-Areas. 
Pro Placement-Area können maximal drei Placements hinzugefügt werden.
Eine gezielte Platzierung der Areas erzwingt strategische Entscheidungen auf Seiten des Spielers.

\subsection{Lane Placement}
Bei einem Lane Placement werden auf einem Weg mehrere Placement-Areas verteilt.
Dies ermöglicht die Aufteilung der Mercenaries in Stopper und Sniper. 
Die Aufgabe der Stopper ist es einen Bossgegner zu schwächen. 
Wohingegen ein Sniper mit wenig Leben aber viel Durchschlagskraft den Boss besiegt.

\subsection{Parallel Placement}
Die Platzierung der Areas auf parallelen Bahnen zwingt den Spieler zu Beginn eine Bahn zu favorisieren oder beide zu vernachlässigen. 
Im Laufe des Spiels wird dieser Nachteil jedoch ausgeglichen.

\section{Unterschiede zwischen den Modi}
Während im normalen Modus eine isometrische Perspektive gewählt wurde, ist der Spieler in der virtuellen Ansicht direkt im Geschehen.
Die isometrische Ansicht bietet eine strategische Übersicht, die schnelle Entscheidungen ermöglicht.
Im direkten Gegensatz dazu muss der Spieler im virtuellen Modus erst die Welt erkunden und herausfinden wo die einzelnen Spawnpunkte bzw. Pfade liegen. 

