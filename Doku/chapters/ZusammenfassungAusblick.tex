\chapter{Zusammenfassung und Ausblick}
Diese Arbeit hat mit You Only Look Once einen Deep Learning Algorithmus f�r die Objekterkennung vorgestellt. 
Die Leistungsf�higkeit von YOLO wird auf Basis eines bin�ren Klassifikators evaluiert. 
Dazu werden relative Kenngr��en in Form der Genauigkeit und Trefferquote bestimmt und in einem Graph dargestellt.
Die Fl�che unterhalb des Graphen ist als Average Precision definiert und wird in einem Mittelwert �ber alle Klassen in der Mean Average Precision zusammengefasst.\newline
Kontr�r zu alternativen Verfahren interpretiert YOLO das Bild als ein Regressions- statt einem Klassifikationsproblem.
YOLO verwendet eine einheitliche Netzarchitektur, die eine Parallelisierung der Prozesse erlaubt.
Daraus ergibt sich eine Verarbeitungsgeschwindigkeit von mehr als 30 Bildern pro Sekunde. 
Das Bild wird in Zellen aufgeteilt und jede Zelle bestimmt eine feste Anzahl von Bounding Boxes. \newline
Im Vergleich mit alternativen Echtzeitverfahren zeigt sich, dass YOLO eine mehr als doppelt so hohe mAP erreicht. 
Eine schlanke Version von YOLO, genannt Fast YOLO, kann 50\% mehr Bilder pro Sekunde verarbeiten als das schnellste alternative Verfahren.
Bei der Fehleranalyse zeigt sich, dass YOLO bei der Lokalisierung schlechter als die Konkurrenz abschneidet.